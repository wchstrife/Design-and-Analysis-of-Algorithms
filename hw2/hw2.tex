%%%%%%%%%%%%%%%%%%%%%%%%%%%%%%%%%%%%%%%%%
% University/School Laboratory Report
% LaTeX Template
% Version 3.1 (25/3/14)
%
% This template has been downloaded from:
% http://www.LaTeXTemplates.com
%
% Original author:
% Linux and Unix Users Group at Virginia Tech Wiki 
% (https://vtluug.org/wiki/Example_LaTeX_chem_lab_report)
%
% License:
% CC BY-NC-SA 3.0 (http://creativecommons.org/licenses/by-nc-sa/3.0/)
%
%%%%%%%%%%%%%%%%%%%%%%%%%%%%%%%%%%%%%%%%%

%----------------------------------------------------------------------------------------
%	PACKAGES AND DOCUMENT CONFIGURATIONS
%----------------------------------------------------------------------------------------

\documentclass[UTF8]{ctexart}

\usepackage{siunitx} % Provides the \SI{}{} and \si{} command for typesetting SI units
\usepackage{graphicx} % Required for the inclusion of images
\usepackage{natbib} % Required to change bibliography style to APA
\usepackage{amsmath} % Required for some math elements 
\usepackage{amssymb} % 使用因为所以符号
\usepackage{fancyhdr} % 使用页眉

\pagestyle{fancy}
\fancyhf{} 
\cfoot{\thepage} 
% TODO 自定义页眉

\setlength\parindent{0pt} % Removes all indentation from paragraphs

\renewcommand{\labelenumi}{\alph{enumi}.} 

%----------------------------------------------------------------------------------------
%	DOCUMENT INFORMATION
%----------------------------------------------------------------------------------------
\title{算法分析与设计-作业二}

\author{王宸昊}

\date{\today}

\begin{document}

\maketitle

%----------------------------------------------------------------------------------------
%	SECTION 1
%----------------------------------------------------------------------------------------

\section{最近点对}

% If you have more than one objective, uncomment the below:
%\begin{description}
%\item[First Objective] \hfill \\
%Objective 1 text
%\item[Second Objective] \hfill \\
%Objective 2 text
%\end{description}

\subsection{算法简介}
\label{definitions}
\begin{description}
\item[Stoichiometry]
The relationship between the relative quantities of substances taking part in a reaction or forming a compound, typically a ratio of whole integers.
\item[Atomic mass]
The mass of an atom of a chemical element expressed in atomic mass units. It is approximately equivalent to the number of protons and neutrons in the atom (the mass number) or to the average number allowing for the relative abundances of different isotopes. 
\end{description} 

\subsection{算法伪代码}

\subsection{实验效果}

\subsection{进一步改进}

\subsection{界面展示}

%----------------------------------------------------------------------------------------
%	SECTION 2
%----------------------------------------------------------------------------------------
\section{三种不同的方法求Fibonacci数}

\subsection{算法简介}
\label{definitions}
\begin{description}
\item[Stoichiometry]
The relationship between the relative quantities of substances taking part in a reaction or forming a compound, typically a ratio of whole integers.
\item[Atomic mass]
The mass of an atom of a chemical element expressed in atomic mass units. It is approximately equivalent to the number of protons and neutrons in the atom (the mass number) or to the average number allowing for the relative abundances of different isotopes. 
\end{description} 

\subsection{算法伪代码}

\subsection{实验效果}

\subsection{进一步改进}


%----------------------------------------------------------------------------------------
%	SECTION 3
%----------------------------------------------------------------------------------------

\section{CLRS, Page, 72 5.3-4}

证明:\\
(1) 根据题意得,当dest = j 时,表示$A[i]$出现在$B[j]$的位置,即当$i+offset = j$ 或 $i+offset-n = j$。根据offset的含义可知,offset只是在原A[i]的数组的基础上循环移位offset位,对于每个i,映射到B中的唯一位置j中。$\therefore$ offset取不同值B[j]的取值不同,A[i]出现在B[j]的概为$\frac{1}{n}$ \\
(2) 此算法没有改变数组元素之间的相对位置,并不能产生均匀随机排列。例如假设$A[]={[1,2,3]}$。随机排列应有$3!=6$种,根据此算法的执行过程,当offset=1时,B[]=[3,1,2];当offset=2时,B[]=[2,3,1];当offset=3时,B[]=[1,2,3]。只能得到3种排列,无法产生类似[2, 1, 3]类似的排列。

%----------------------------------------------------------------------------------------
%	SECTION 4
%----------------------------------------------------------------------------------------

\section{CLRS, Page, 73 5.3-5}

证明: \\
设$X_i$表示第i个位置元素唯一,共有$n^3$个元素。 \\
\begin{align*}
    P(\text{所有元素都唯一}) &= P(X_1\bigcap X_2\bigcap X_3 \cdots \bigcap X_n) \\
        &= 1(1-\frac{1}{n^3})(1-\frac{2}{n^3})(1-\frac{3}{n^3})\cdots(1-\frac{n}{n^3})\\
        &\geq 1(1-\frac{n}{n^3})(1-\frac{n}{n^3})(1-\frac{n}{n^3})\cdots(1-\frac{n}{n^3})\\
        &\geq (1-\frac{1}{n^2})^n\\
        &\geq (1-\frac{1}{n})
\end{align*}

\end{document}