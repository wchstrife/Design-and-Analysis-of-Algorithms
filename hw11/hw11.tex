%%%%%%%%%%%%%%%%%%%%%%%%%%%%%%%%%%%%%%%%%
% University/School Laboratory Report
% LaTeX Template
% Version 3.1 (25/3/14)
%
% This template has been downloaded from:
% http://www.LaTeXTemplates.com
%
% Original author:
% Linux and Unix Users Group at Virginia Tech Wiki 
% (https://vtluug.org/wiki/Example_LaTeX_chem_lab_report)
%
% License:
% CC BY-NC-SA 3.0 (http://creativecommons.org/licenses/by-nc-sa/3.0/)
%
%%%%%%%%%%%%%%%%%%%%%%%%%%%%%%%%%%%%%%%%%

%----------------------------------------------------------------------------------------
%	PACKAGES AND DOCUMENT CONFIGURATIONS
%----------------------------------------------------------------------------------------

\documentclass[UTF8]{ctexart}

\usepackage{siunitx} % Provides the \SI{}{} and \si{} command for typesetting SI units
\usepackage{graphicx} % Required for the inclusion of images
\usepackage{graphics} % 图片设置
\usepackage{subfigure} 

\usepackage{natbib} % Required to change bibliography style to APA
\usepackage{amsmath} % Required for some math elements 
\usepackage{amssymb} % 使用因为所以符号
\usepackage{fancyhdr} % 使用页眉

\usepackage{algorithm}
\usepackage{algorithmic}

\usepackage{listings} % 插入代码
\usepackage{xcolor}

\usepackage{enumerate} % 列表

\lstset{
    %backgroundcolor=\color{red!50!green!50!blue!50},%代码块背景色为浅灰色
    rulesepcolor= \color{gray}, %代码块边框颜色
    breaklines=true,  %代码过长则换行
    numbers=left, %行号在左侧显示
    numberstyle= \small,%行号字体
    %keywordstyle= \color{red},%关键字颜色
    commentstyle=\color{gray}, %注释颜色
    frame=shadowbox%用方框框住代码块
    }

%\usepackage{url} % 引用URL
% \usepackage{cite}
% \newcommand{\upcite}[1]{\textsuperscript{\textsuperscript{\cite{#1}}}} %参考文献上标
%\bibliographystyle{plain}   %引用的样式%

\pagestyle{fancy}
\fancyhf{} 
\cfoot{\thepage} 

\setlength\parindent{0pt} % Removes all indentation from paragraphs

\renewcommand{\labelenumi}{\alph{enumi}.} 

%----------------------------------------------------------------------------------------
%	DOCUMENT INFORMATION
%----------------------------------------------------------------------------------------
\title{算法分析与设计-作业十一}

\author{王宸昊 2019214541}

\date{\today}

\begin{document}

\maketitle

%----------------------------------------------------------------------------------------
%	SECTION 1
%----------------------------------------------------------------------------------------

\section{CLRS, Page,647 34.5-7}

给定一个图G和一个路径p,以路径p为证书。判断某个路径是否是图G的简单回路,可以通过以下条件进行判断:\\
1. p中每一条边是否在图G上\\
2. 路径p的起始点和终止点是否相同\\
3. 路径除了起点和终点外,其他的点是否不重复\\
4. 路径长度是否为K\\
通过以上的判断,可以在$O(n)$时间内验证,所以该问题是NP问题。\\

规约过程选取哈密尔顿回路问题。给定一个哈密尔顿回路的实例,即图$G=(V,E)$,进行判断路径$p_1$是否为哈密尔顿回路。将图复制为$G’=(V,E)$,G‘与$p_1$对应的回路是$p_2$,通过判断$p_2$是否是图G'的长度为$|V|$的简单回路,即可判断出$p_1$是否是图G的哈密尔顿回路,这样就将问题转化为了已知的问题。\\
而将图复制为G'时间复杂度为$O(V + E)$,属于多项式时间。\\

综上该问题属于NP问题。
%----------------------------------------------------------------------------------------
%	SECTION 2
%----------------------------------------------------------------------------------------

\section{CLRS, Page, 660 35.3-3}

由于在贪心近似算法中,每一次循环都能选择出覆盖最多尚未被覆盖元素的集合S,所以在每次选择的时候,需要按一下方法制定选择的规则。

(1).遍历$S\in F$,建立一个点-集合的映射关系map,即$map[p]={S_1,S_2,...}$,其中$p\in S_1,p\in S_2,...$,同时得到每个集合$S\in F$覆盖点的大小$S.length$。所花时间为$O(\Sigma_{S\in F}|S|)$。

(2)通过找到最大的$S.length$来得到最大的$S\cap U$。所花时间为$O(|X|*|F|)$。

(3)对选择出的$S$,对每个在$U$中的$p\in S$,操作$U-{p}$,然后遍历$s \in map[p]$,让$s.length-=1$。所花时间为$O(|X|*|F|)$。

该算法所花时间为$O(\Sigma_{S\in F}|S|)$。




\end{document}
