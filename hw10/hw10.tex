%%%%%%%%%%%%%%%%%%%%%%%%%%%%%%%%%%%%%%%%%
% University/School Laboratory Report
% LaTeX Template
% Version 3.1 (25/3/14)
%
% This template has been downloaded from:
% http://www.LaTeXTemplates.com
%
% Original author:
% Linux and Unix Users Group at Virginia Tech Wiki 
% (https://vtluug.org/wiki/Example_LaTeX_chem_lab_report)
%
% License:
% CC BY-NC-SA 3.0 (http://creativecommons.org/licenses/by-nc-sa/3.0/)
%
%%%%%%%%%%%%%%%%%%%%%%%%%%%%%%%%%%%%%%%%%

%----------------------------------------------------------------------------------------
%	PACKAGES AND DOCUMENT CONFIGURATIONS
%----------------------------------------------------------------------------------------

\documentclass[UTF8]{ctexart}

\usepackage{siunitx} % Provides the \SI{}{} and \si{} command for typesetting SI units
\usepackage{graphicx} % Required for the inclusion of images
\usepackage{graphics} % 图片设置
\usepackage{subfigure} 

\usepackage{natbib} % Required to change bibliography style to APA
\usepackage{amsmath} % Required for some math elements 
\usepackage{amssymb} % 使用因为所以符号
\usepackage{fancyhdr} % 使用页眉

\usepackage{algorithm}
\usepackage{algorithmic}

\usepackage{listings} % 插入代码
\usepackage{xcolor}

\usepackage{enumerate} % 列表

\lstset{
    %backgroundcolor=\color{red!50!green!50!blue!50},%代码块背景色为浅灰色
    rulesepcolor= \color{gray}, %代码块边框颜色
    breaklines=true,  %代码过长则换行
    numbers=left, %行号在左侧显示
    numberstyle= \small,%行号字体
    %keywordstyle= \color{red},%关键字颜色
    commentstyle=\color{gray}, %注释颜色
    frame=shadowbox%用方框框住代码块
    }

%\usepackage{url} % 引用URL
% \usepackage{cite}
% \newcommand{\upcite}[1]{\textsuperscript{\textsuperscript{\cite{#1}}}} %参考文献上标
%\bibliographystyle{plain}   %引用的样式%

\pagestyle{fancy}
\fancyhf{} 
\cfoot{\thepage} 

\setlength\parindent{0pt} % Removes all indentation from paragraphs

\renewcommand{\labelenumi}{\alph{enumi}.} 

%----------------------------------------------------------------------------------------
%	DOCUMENT INFORMATION
%----------------------------------------------------------------------------------------
\title{算法分析与设计-作业十}

\author{王宸昊 2019214541}

\date{\today}

\begin{document}

\maketitle

%----------------------------------------------------------------------------------------
%	SECTION 1
%----------------------------------------------------------------------------------------

\section{CLRS, Page,594 32.4-8}




%----------------------------------------------------------------------------------------
%	SECTION 2
%----------------------------------------------------------------------------------------

\section{CLRS, Page, 594 32-1}




%----------------------------------------------------------------------------------------
%	SECTION 3
%----------------------------------------------------------------------------------------
\section{CLRS, Page, 447 26-2}

\subsection{实验环境:}

编译环境为C++10,操作系统的版本为Windows 10。GUI框架选用QT5\\
硬件环境为:CPU为AMD Ryzen 7 1700(3.0 GHz),RAM 16G。\\

\subsection{实验运行:}

可执行EXE文件位于src目录下,已打包好可直接运行。\\

编译源码请使用Qt Creator打开.user文件,在Qt Creator中编译执行。

\subsection{实验原理:}

\subsubsection{Brute-Force算法}

将匹配串和模式串左端对齐,依次跟模式串的每一位比较是否一致,如果不一致将匹配串右移一位。将匹配中的首位位置记录下来。\\

同时暴力法不需要预处理。。假设匹配串的长度为m, 模式串的长度为n, 则算法的复杂度为$O(mn)$。

\subsubsection{KMP算法}

KMP算法的基本思想是主串不回溯,通过计算前缀数组,每次不匹配时,不需要每次回退主串的下标,而且通过前缀数组记录的信息进行重新对齐。\\

预处理操作包括计算前缀数组,同时还需要$O(n)$的空间,算法的复杂度为$O(m)$


\subsubsection{BM算法}


\subsection{实验结果:}

测试数据集共设计了两种。位于data目录下。\\

数据集1是原版的哈利波特的TXT文件,每个文件包含几十万个英文数字字符和标点符号。\\
数据集2是Python脚本随机生成的字符文本,字符集合包括英文字母数字和特殊符号,数据集大小在两百万。\\

\end{document}
