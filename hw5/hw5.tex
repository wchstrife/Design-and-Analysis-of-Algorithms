%%%%%%%%%%%%%%%%%%%%%%%%%%%%%%%%%%%%%%%%%
% University/School Laboratory Report
% LaTeX Template
% Version 3.1 (25/3/14)
%
% This template has been downloaded from:
% http://www.LaTeXTemplates.com
%
% Original author:
% Linux and Unix Users Group at Virginia Tech Wiki 
% (https://vtluug.org/wiki/Example_LaTeX_chem_lab_report)
%
% License:
% CC BY-NC-SA 3.0 (http://creativecommons.org/licenses/by-nc-sa/3.0/)
%
%%%%%%%%%%%%%%%%%%%%%%%%%%%%%%%%%%%%%%%%%

%----------------------------------------------------------------------------------------
%	PACKAGES AND DOCUMENT CONFIGURATIONS
%----------------------------------------------------------------------------------------

\documentclass[UTF8]{ctexart}

\usepackage{siunitx} % Provides the \SI{}{} and \si{} command for typesetting SI units
\usepackage{graphicx} % Required for the inclusion of images
\usepackage{graphics} % 图片设置
\usepackage{subfigure} 

\usepackage{natbib} % Required to change bibliography style to APA
\usepackage{amsmath} % Required for some math elements 
\usepackage{amssymb} % 使用因为所以符号
\usepackage{fancyhdr} % 使用页眉

\usepackage{algorithm}
\usepackage{algorithmic}

\usepackage{listings} % 插入代码
\usepackage{xcolor}

\usepackage{enumerate} % 列表

\lstset{
    %backgroundcolor=\color{red!50!green!50!blue!50},%代码块背景色为浅灰色
    rulesepcolor= \color{gray}, %代码块边框颜色
    breaklines=true,  %代码过长则换行
    numbers=left, %行号在左侧显示
    numberstyle= \small,%行号字体
    %keywordstyle= \color{red},%关键字颜色
    commentstyle=\color{gray}, %注释颜色
    frame=shadowbox%用方框框住代码块
    }

%\usepackage{url} % 引用URL
% \usepackage{cite}
% \newcommand{\upcite}[1]{\textsuperscript{\textsuperscript{\cite{#1}}}} %参考文献上标
%\bibliographystyle{plain}   %引用的样式%

\pagestyle{fancy}
\fancyhf{} 
\cfoot{\thepage} 

\setlength\parindent{0pt} % Removes all indentation from paragraphs

\renewcommand{\labelenumi}{\alph{enumi}.} 

%----------------------------------------------------------------------------------------
%	DOCUMENT INFORMATION
%----------------------------------------------------------------------------------------
\title{算法分析与设计-作业五}

\author{王宸昊 2019214541}

\date{\today}

\begin{document}

\maketitle

%----------------------------------------------------------------------------------------
%	SECTION 1
%----------------------------------------------------------------------------------------

\section{CLRS, Page, 241 16.1-2}

\subsection{算法设计}
首先先对活动按照开始的时间$s_i$从晚到早进行排序,每次优先选择开始时间最晚的活动加入任务集合,且该活动必须与任务集合当中的活动相容。如果有最晚开始时间相同的,则任意选择一个。\\
例如迭代式的贪心算法:假定输入活动已按开始时间单调递减的顺序排好序。它将选出的活动存入集合A。
\begin{enumerate}[1.]

    \item 首先将第一个活动加入A当中。
    \item 遍历输入活动中的每个活动,如果结束时间早于A当中末尾活动的开始时间,则加入集合A。
    \item 返回集合A,即为一个最大相容的活动集合。
    
\end{enumerate}

\subsection{证明}
令$A_i$是最大兼容活动子集,$S_i$是活动$a_i$开始之前结束的活动集合。\\
\textbf{活动选择的最优子结构:}\\
假设$a_j \in A_i$且$a_j$是其中最晚开始的活动,设$a_j \in A_i$是$A_i$里最晚开始的活动,A{i}必然包含子问题$S_{i-1}=S_{i}-{a_j}$的最优解.\\
这里使用剪切-粘贴法证明:如果可以找到$S_{i-1}$的一个相容活动子集$A^\prime_{i-1}$满足$|A^\prime_{i-1}|>|A_{i-1}|$,则可以将$A^\prime_{i-1}$而不是$A_{i-1}$作为$S_i$的最优解的一部分.这样就构造出一个兼容活动集,其大小比$A_i$更大,与假设矛盾,故最优子结构得证。\\
\textbf{贪心选择的正确性:}\\
即证明对于$S_k$,令$a_m$是$S_k$中开始时间最晚的活动,则$a_m$在$S_k$的某个最大相容活动子集中。\\
令$A_k$是$S_k$的一个最大相容活动子集,且$a_j$是$A_k$中开始时间最晚的活动.若$a_j=a_m$,则已经证明$a_m$在$S_k$的某个最大兼容活动子集中;否则,令集合$A^\prime_k=A_k-{a_j}\cup {a_m}$,即将$A_k$中的$a_j$替换成$a_m$.由于$s_m>=s_i$,可知$A^\prime_k$中活动都相容,且$|A^\prime_k|=|A_k|$,因此得出$A^\prime_k$也是$S_k$的一个最大相容活动子集,且包含$a_m$.故得证。

%----------------------------------------------------------------------------------------
%	SECTION 2
%----------------------------------------------------------------------------------------

\section{CLRS, Page, 245 16.2-6}

\subsection{算法设计}
首先计算每个商品的每磅价值$v_i / w_i$,并寻找到中间的价值记做$m$。此时将所有的商品分为三种情况:第一种是每磅价值大于m的,记做G;第二种是每磅价值等于m的,记做E;第三种是每磅价值小于m的,记做L。分别将3种类别当中的重量和计算出来,记做$W_g, W_e, W_m$。以上操作都在线性时间内完成。递归方法的输入为当前背包可承受的重量为W和备选物品集合,下面对3种子问题进行讨论:
\begin{enumerate}[1.]

    \item 若$W_G + W_E < W$,则递归调用,子问题的背包容量为$W-W_G-W_E$,备选物品集合为L
    \item 若$W_G > W$,则递归调用,子问题背包容量为W,备选物品集合为$W_G$
    \item 若$W_G <= W <= W_G + W_E$,则将G当中的商品全部加入,剩下的$W-W_G$全部装E中的商品

\end{enumerate}

根据上面的过程,可以的出递归的递推式为$T(n) = T(n/2) + O(n)$, 根据主方法可以得到复杂度为O(n)。

\subsection{伪代码}

\begin{lstlisting}
    def fractional_knapsack(total_weight, items_list):
        if items_list.empty():
            return 0
        G, E, L = [], [], []
        m = calculate_value(items_list)
        for var in items_list:
            if var.value > m:
                G.append(var)
            elif var.value == m:
                E.append(var)
            else:
                L.append(var)

        W_G = calculate_total_weight(G)
        W_E = calculate_total_weight(E)
        V_G = calculate_total_value(G)
        V_E = calculate_total_value(E)
        if total_weight > W_E+W_G:
            return V_G+V_E+fractional_knapsack(total_weight-W_E-W_G, L)
        elif total_weight < W_G:
            return fractional_knapsack(total_weight, G)
        else:
            return V_G+(total_weight-W_G)*m
 \end{lstlisting}
  
%----------------------------------------------------------------------------------------
%	SECTION 3
%----------------------------------------------------------------------------------------

\section{CLRS, Page, 436 16.3-7}

\subsection{算法设计}
实现三进制huffman编码,只需要在2进制的基础上,用一棵三叉树来表示。\\
算法的伪代码如下:

\begin{lstlisting}
    def HUFFMAN_TRI(C):
        n = len(C)
        Q = C
        for i in range(n)
            allocate a new node z
            z.left = x = EXTRACT-MIN(Q)
            z.mid = m = EXTRACT-MIN(Q)
            z.right = y = EXTRACT-MIN(Q)
            INSERT(Q, z)
        return EXTRACT-MIN(Q)
 \end{lstlisting}

\subsection{证明}
为了证明HUFFMAN三进制的正确性,需要证明确定三进制最优前缀码的问题具有贪心选择和最优子结构性质。\\

\textbf{贪心选择性质:}\\
令$C$为一个字母表,其中每个字符$c\in C$都有一个频率$c.freq$.
令$x$和$y$和$w$是$C$中频率最低的三个字符,那么存在$C$的一个最优前缀码,使得
$x$和$y$和$w$的码字长度相同,且只有最后一个三进制位不同.

证明:令$a$和$b$和$d$是$T$中深度最大的兄弟叶节点.假定$a.freq<=b.freq<=d.freq$且
$x.freq<=y.freq<=w.freq$.由于$x.freq$和$y.freq$和$w.freq$是是叶节点中最低的
三个频率,而$a.freq,b.freq,d.freq$是三个任意的频率,因此有$x.freq<=a.freq,
y.freq<=b.freq,w.freq<=d.ferq$.

假设在$T$中交换$x$和$a$生成一颗新树$T^\prime$,那么$T$和$T^\prime$的代价差为

\begin{equation*}
	\begin{aligned}
        B(T)-B(T^\prime) &= x.freq*d_T(x)+a.freq*d_T(a)-x.freq*d_{T^\prime}(x)-x.freq*d_{T^\prime}(a) \\
                        &= x.freq*d_T(x)+a.freq*d_T(a)-x.freq*d_{T}(x)-x.freq*d_{T}(a) \\
                        &= (a.freq-x.freq)(d_T(a)-d_T(x)) \\
                        &>=0
	\end{aligned}
\end{equation*}

同理,在$T^\prime$中交换$y$和$b$得到$T^{\prime \prime}$,在$T^{\prime \prime}$
中交换$w$和$d$得到$T^{\prime \prime \prime}$,得知$B(T)>=B(T^{\prime})>=B(T^{\prime \prime})>=B(T^{\prime \prime \prime})$.
又由于$T$是最优的,有$B(T)<=B(T^{\prime \prime \prime})$.
所以$B(T)=B(T^{\prime \prime \prime})$,故$T^{\prime \prime \prime}$也是最优树,
且$x$,$y$,$z$是深度最深的兄弟叶节点。\\

\textbf{最优子结构:}\\
令C为一个给定的字母表,其中每个字符$c\in C$都定义了一个频率$c.freq$.
令$x$和$y$和$w$是$C$中频率最低的三个字符,令$C^\prime$为$C$去掉字符$x$和$y$和$w$,加入新字符
$z$后得到的字母表,定义$C^\prime$的频率与$C$相同,不同之处在于$z.freq=x.freq+y.freq+w.freq$.
令$T^\prime$为字母表$C^\prime$的一个最优前缀码对应的编码树.当$T^\prime$中叶节点$z$
替换为一个以$x$和$y$和$w$为孩子的节点,得到的树$T$,$T$表示的就是字母表$C$的一个最优前缀码.

证明:设$d_T(x)$为在$T$这棵树上,$x$元素离根节点的距离,$B(T)$表示树$T$的总代价.
由题目可得$B(T)=B(T^\prime)+x.freq+y.freq+w.freq$.

假设$T$对应的前缀码不是$C$的最优前缀码,那么存在一个最优编码树$T^{\prime \prime}$,使得
$B(T^{\prime\prime})<B(T)$.令$T^{\prime \prime \prime}$为将$T^{\prime \prime}$
中的$x$,$y$,$z$及它们的父节点替换为叶节点$z$得到的树,其中$z.freq=x.freq+y.freq+w.freq$.
于是:

\begin{equation*}
	\begin{aligned}
        B(T^{\prime \prime \prime}) &= B(T^{\prime \prime}) - x.freq-y.freq-w.freq \\
                                    &<B(T)-x.freq-y.freq-w.freq \\
                                    &=B(T^{\prime})
	\end{aligned}
\end{equation*}
与$T^{\prime}$是最优编码树的假设矛盾.故$T$必然表示字母表$C$的一个最优前缀编码.

\end{document}