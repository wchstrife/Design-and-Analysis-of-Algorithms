%%%%%%%%%%%%%%%%%%%%%%%%%%%%%%%%%%%%%%%%%
% University/School Laboratory Report
% LaTeX Template
% Version 3.1 (25/3/14)
%
% This template has been downloaded from:
% http://www.LaTeXTemplates.com
%
% Original author:
% Linux and Unix Users Group at Virginia Tech Wiki 
% (https://vtluug.org/wiki/Example_LaTeX_chem_lab_report)
%
% License:
% CC BY-NC-SA 3.0 (http://creativecommons.org/licenses/by-nc-sa/3.0/)
%
%%%%%%%%%%%%%%%%%%%%%%%%%%%%%%%%%%%%%%%%%

%----------------------------------------------------------------------------------------
%	PACKAGES AND DOCUMENT CONFIGURATIONS
%----------------------------------------------------------------------------------------

\documentclass[UTF8]{ctexart}

\usepackage{siunitx} % Provides the \SI{}{} and \si{} command for typesetting SI units
\usepackage{graphicx} % Required for the inclusion of images
\usepackage{graphics} % 图片设置
\usepackage{subfigure} 

\usepackage{natbib} % Required to change bibliography style to APA
\usepackage{amsmath} % Required for some math elements 
\usepackage{amssymb} % 使用因为所以符号
\usepackage{fancyhdr} % 使用页眉

\usepackage{algorithm}
\usepackage{algorithmic}

\usepackage{listings} % 插入代码
\usepackage{xcolor}

\usepackage{enumerate} % 列表

\lstset{
    %backgroundcolor=\color{red!50!green!50!blue!50},%代码块背景色为浅灰色
    rulesepcolor= \color{gray}, %代码块边框颜色
    breaklines=true,  %代码过长则换行
    numbers=left, %行号在左侧显示
    numberstyle= \small,%行号字体
    %keywordstyle= \color{red},%关键字颜色
    commentstyle=\color{gray}, %注释颜色
    frame=shadowbox%用方框框住代码块
    }

%\usepackage{url} % 引用URL
% \usepackage{cite}
% \newcommand{\upcite}[1]{\textsuperscript{\textsuperscript{\cite{#1}}}} %参考文献上标
%\bibliographystyle{plain}   %引用的样式%

\pagestyle{fancy}
\fancyhf{} 
\cfoot{\thepage} 

\setlength\parindent{0pt} % Removes all indentation from paragraphs

\renewcommand{\labelenumi}{\alph{enumi}.} 

%----------------------------------------------------------------------------------------
%	DOCUMENT INFORMATION
%----------------------------------------------------------------------------------------
\title{算法分析与设计-作业六}

\author{王宸昊 2019214541}

\date{\today}

\begin{document}

\maketitle

%----------------------------------------------------------------------------------------
%	SECTION 1
%----------------------------------------------------------------------------------------

\section{CLRS, Page, 448 16-4}

\subsection{证明:此算法总能得到最优解}

此算法的思想是将最小化任务惩罚等价于最大化提前任务惩罚,假设集合O是该算法得到的最优解,当处理任务$a_j$时共有三种情况。

\begin{enumerate}[1.]
    \item 如果$a_j$的对应空时间槽在$d_j$之前,那么$a_j$和在$d_j$的活动进行交换,因为此时处于$d_j$处的活动的惩罚一定大于当前活动的惩罚,而且没有超过其结束时间,所以此交换不增加惩罚。
    \item 假如$a_j$的对应空时间槽在$d_j$之后,在$a_j$之前的活动的惩罚都大于$w_j$,所以此时做交换是增加惩罚的,只能将其作为延迟任务。
    \item 假如$a_j$的对应空时间槽在$d_j$之后,那么一定存在$k>j$,$a_k$的位置位于$d_j$之前,那么交换$a_j$和$a_k$能减少惩罚,但是由于集合O对应的是在最优解,这种情况并不成立。
\end{enumerate}

综合以上分析,贪心策略可以得到最优解。

\subsection{算法实现}
不相交集合森林用于处理元素不相交的集合,在该算法当中,我们共有两个集合,分别是可以完成的任务和不可完成的任务的集合。使用不相交森林查询当前活动属于哪个集合。\\
$MAKE-SET(x)$返回当前x属于的集合,用一个数组$D[i]$表示当前活动$d_i$所占用的时间范围.\\

伪代码如下:

\begin{lstlisting}
    SCHEDUING-DISJOINT-SET-FOREST(A)
    init an array D of size A 
    for i = 1 to n
        a[i].time = a[i].deadline
        if D[a[i].deadline] != NULL
            y = FIND-SET(D[a[i].deadline])
            a[i].time = y.low - 1
        x = MAKE-SET(a[i])
        D[a[i].time] = x
        x.low = x.high = a[i].time
        if D[a[i].time − 1] != NULL
            UNION(D[a[i].time − 1] , D[a[i].time])
        if D[a[i]. time + 1] != NULL
            UNION(D[a[i].time], D[a[i].time + 1])
 \end{lstlisting}

 根据21.3中的复杂度分析,使用了不相交森林的方法后,时间复杂度从$O(n^2)$降低到$O(ma(n))$,其中$a(n)$是一个增长很慢的函数,可以证明$a(n) <= 4$
%----------------------------------------------------------------------------------------
%	SECTION 2
%----------------------------------------------------------------------------------------

\section{CLRS, Page, 270 17.4-3}
假设本次Delete时,$a_{i-1} > \frac{1}{3}$,则不会产生表的收缩,则第i个操作的摊还代价为:

\begin{align*}
    {\widehat{c}}_i &=  C_i + {\Phi}_i - {\Phi}_{i-1} \\
                    &= 1 + | 2\cdot num_i - size_i | - | 2\cdot num_{i-1} - size_{i-1} |\\
                    &= 1 + | 2\cdot num_i - size_i | - | 2\cdot num_{i-1} - size_{i} |\\
                    &= 3           
\end{align*}

当$a_{i-1} \leq \frac{1}{3}$时,此时引起了收缩.
\begin{align*}
    {\widehat{c}}_i &=  C_i + {\Phi}_i - {\Phi}_{i-1} \\
                    &= num_i + 1 + | 2\cdot num_i - size_i | - | 2\cdot num_{i-1} - size_{i-1} |\\
                    &= num_i + 1 + (\frac{2}{3}size_{i-1} - 2\cdot num_i) - (size_{i-1} - 2\cdot num_i - 2)\\
                    &= num_i + 1 + (2\cdot num_i + 2 - 2\cdot num_i) - (3\cdot num_i + 3 - 2\cdot num_i)\\
                    &= 2         
\end{align*}
因此该操作的摊还代价的上界是一个常数。  

%----------------------------------------------------------------------------------------
%	SECTION 3
%----------------------------------------------------------------------------------------

\section{CLRS, Page, 270 17-2}

\subsection{SEARCH操作}
依次遍历每一个数组,在每个数组内进行二分查找。
\begin{align*}
    T(n)    &=  \Theta  (\sum_{i = 0}^{i = k-1}lg2^i )  \\  
            &=  \Theta  (\sum_{i = 0}^{i = k-1}i ) \\
            &= \Theta  (k(k-1)/2)\\
            &= \Theta({lg}^2n)
\end{align*}

\subsection{INSERT操作}
将insert的数与$A_0$当中的数进行插入,依次会后面数组中的数,假如$n_i=1$,表示第i个数组中满元素,将上一层的数组进行排序,插入下一层数组中,依次进行类似的操作,直到$n_j = 0$,将前面层的元素全部插入到第j个数组中。对于每个数组的有序化是线性时间的。
因此在最坏的情况下,有n个数字需要有序的插入,因此最坏的复杂度为O(n).\\

均摊分析:\\
使用accounting method进行摊还分析,在每次插入的时,每个元素额外将lgn的花费存为信用,用于后续的排序操作,由于总共有lgn个数组,最多进行lgn次所以在插入的过程中,不会出现“欠债”的现象,所以lgn可以作为均摊分析的一个上界:O(lgn)

\subsection{DELETE操作}
找到第一个$n_i=1$的数组,将要删除的元素和$A_i$当中的任意一个元素替换,然后将$A_i$依次填入到$A_0$到$A_{i-1}$中。


\end{document}