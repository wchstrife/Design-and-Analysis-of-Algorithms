%%%%%%%%%%%%%%%%%%%%%%%%%%%%%%%%%%%%%%%%%
% University/School Laboratory Report
% LaTeX Template
% Version 3.1 (25/3/14)
%
% This template has been downloaded from:
% http://www.LaTeXTemplates.com
%
% Original author:
% Linux and Unix Users Group at Virginia Tech Wiki 
% (https://vtluug.org/wiki/Example_LaTeX_chem_lab_report)
%
% License:
% CC BY-NC-SA 3.0 (http://creativecommons.org/licenses/by-nc-sa/3.0/)
%
%%%%%%%%%%%%%%%%%%%%%%%%%%%%%%%%%%%%%%%%%

%----------------------------------------------------------------------------------------
%	PACKAGES AND DOCUMENT CONFIGURATIONS
%----------------------------------------------------------------------------------------

\documentclass[UTF8]{ctexart}

\usepackage{siunitx} % Provides the \SI{}{} and \si{} command for typesetting SI units
\usepackage{graphicx} % Required for the inclusion of images
\usepackage{graphics} % 图片设置
\usepackage{subfigure} 

\usepackage{natbib} % Required to change bibliography style to APA
\usepackage{amsmath} % Required for some math elements 
\usepackage{amssymb} % 使用因为所以符号
\usepackage{fancyhdr} % 使用页眉

\usepackage{algorithm}
\usepackage{algorithmic}

\usepackage{listings} % 插入代码
\usepackage{xcolor}

\usepackage{enumerate} % 列表

\lstset{
    %backgroundcolor=\color{red!50!green!50!blue!50},%代码块背景色为浅灰色
    rulesepcolor= \color{gray}, %代码块边框颜色
    breaklines=true,  %代码过长则换行
    numbers=left, %行号在左侧显示
    numberstyle= \small,%行号字体
    %keywordstyle= \color{red},%关键字颜色
    commentstyle=\color{gray}, %注释颜色
    frame=shadowbox%用方框框住代码块
    }

%\usepackage{url} % 引用URL
% \usepackage{cite}
% \newcommand{\upcite}[1]{\textsuperscript{\textsuperscript{\cite{#1}}}} %参考文献上标
%\bibliographystyle{plain}   %引用的样式%

\pagestyle{fancy}
\fancyhf{} 
\cfoot{\thepage} 

\setlength\parindent{0pt} % Removes all indentation from paragraphs

\renewcommand{\labelenumi}{\alph{enumi}.} 

%----------------------------------------------------------------------------------------
%	DOCUMENT INFORMATION
%----------------------------------------------------------------------------------------
\title{算法分析与设计-作业七}

\author{王宸昊 2019214541}

\date{\today}

\begin{document}

\maketitle

%----------------------------------------------------------------------------------------
%	SECTION 1
%----------------------------------------------------------------------------------------

\section{CLRS, Page, 360 22.5-7}

\subsection{算法思想}

首先对有向图G(V, E)使用求强连通分量的算法,,然后对强连通分量进行收缩,形成无环分量图。再对此无环分量图进行拓扑排序,得到的顶点序列,如果依次存在边相连,则该有向图G是半连通的。

\subsection{算法伪代码}

\begin{algorithm}[H]
	\caption{IS-SEMI-CONNECTED(G)}  % 算法标题
    \begin{algorithmic}[1]  % 一行一个标行号
        \STATE Computer the component graph of G, called G'
        \STATE Call topological sort on G', get list of vertices $v_1, v_2, \dots, v_k$.
        \FOR{$i=0$ to $k-1$}
        \IF{no edge between $v_i$ and $v_{i+1}$}
        \RETURN{FALSE}
        \ENDIF
        \ENDFOR
        \RETURN{TRUE}
	\end{algorithmic}
\end{algorithm}

 \subsection{正确性证明}

 假设图G中的任意两个顶点$v_i, v_j$.\\
 当$v_i, v_j$属于同一个强连通分量时,则一定满足半连通的条件。\\
 当$v_i, v_j$不属于同一个强连通分量时,则在对应的无环连通图中,属于不同的顶点,当进行拓扑排序后,如果每个相邻的顶点都存在边,则一定一条路径使得$v_i$可以抵达$v_j$或者$v_j$可以抵达$v_i$,则此时图G是半连通的,否则则不存在$v_i$到$v_j$的边,即不连通。

 \subsection{复杂性分析}
 求强连通分量的时间复杂度为O(V+R),拓扑排序的时间复杂度为O(V+E),因此总的算法复杂度为O(V+E)。

%----------------------------------------------------------------------------------------
%	SECTION 2
%----------------------------------------------------------------------------------------

\section{CLRS, Page, 370 23.2-7}

假设图G(V, E)的最小生成树为T,最小生成树的边集为$E_T$,加入一个新顶点后的最小生成树为T',其相关的边集为$E_{add}$。由树的定义可以知道,树T中边的个数为$V.num-1$,\\
当新加入的结点只有一条边时,即更新后的最小生成树T'就是在T的基础上加上顶点和这条相关的边。
因为此时只有一条边可以抵达该结点,所以此时T'的边的权重一定是最小的。\\
当新加入的结点相关边为k时(k>1),则要在边集\{ $E_{T} \cup E_{add}$ \} 中删除k-1条边。具体删除的策略为对每次顶点进行k-1次DFS,每次都能找到一条回路,然后删除回路中权值最大的一条边。\\
因为在一个有向图中




%----------------------------------------------------------------------------------------
%	SECTION 3
%----------------------------------------------------------------------------------------

\section{CLRS, Page, 370 23-1}

\subsection{SEARCH操作}
依次遍历每一个数组,在每个数组内进行二分查找。
\begin{align*}
    T(n)    &=  \Theta  (\sum_{i = 0}^{i = k-1}lg2^i )  \\  
            &=  \Theta  (\sum_{i = 0}^{i = k-1}i ) \\
            &= \Theta  (k(k-1)/2)\\
            &= \Theta({lg}^2n)
\end{align*}

\subsection{INSERT操作}
将insert的数与$A_0$当中的数进行插入,依次会后面数组中的数,假如$n_i=1$,表示第i个数组中满元素,将上一层的数组进行排序,插入下一层数组中,依次进行类似的操作,直到$n_j = 0$,将前面层的元素全部插入到第j个数组中。对于每个数组的有序化是线性时间的。
因此在最坏的情况下,有n个数字需要有序的插入,因此最坏的复杂度为O(n).\\

均摊分析:\\
使用accounting method进行摊还分析,在每次插入的时,每个元素额外将lgn的花费存为信用,用于后续的排序操作,由于总共有lgn个数组,最多进行lgn次所以在插入的过程中,不会出现“欠债”的现象,所以lgn可以作为均摊分析的一个上界:O(lgn)

\subsection{DELETE操作}
找到第一个$n_i=1$的数组,将要删除的元素和$A_i$当中的任意一个元素替换,然后将$A_i$依次填入到$A_0$到$A_{i-1}$中。


\end{document}