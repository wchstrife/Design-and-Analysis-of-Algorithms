%%%%%%%%%%%%%%%%%%%%%%%%%%%%%%%%%%%%%%%%%
% University/School Laboratory Report
% LaTeX Template
% Version 3.1 (25/3/14)
%
% This template has been downloaded from:
% http://www.LaTeXTemplates.com
%
% Original author:
% Linux and Unix Users Group at Virginia Tech Wiki 
% (https://vtluug.org/wiki/Example_LaTeX_chem_lab_report)
%
% License:
% CC BY-NC-SA 3.0 (http://creativecommons.org/licenses/by-nc-sa/3.0/)
%
%%%%%%%%%%%%%%%%%%%%%%%%%%%%%%%%%%%%%%%%%

%----------------------------------------------------------------------------------------
%	PACKAGES AND DOCUMENT CONFIGURATIONS
%----------------------------------------------------------------------------------------

\documentclass[UTF8]{ctexart}

\usepackage{siunitx} % Provides the \SI{}{} and \si{} command for typesetting SI units
\usepackage{graphicx} % Required for the inclusion of images
\usepackage{graphics} % 图片设置
\usepackage{subfigure} 

\usepackage{natbib} % Required to change bibliography style to APA
\usepackage{amsmath} % Required for some math elements 
\usepackage{amssymb} % 使用因为所以符号
\usepackage{fancyhdr} % 使用页眉

\usepackage{algorithm}
\usepackage{algorithmic}

\usepackage{listings} % 插入代码
\usepackage{xcolor}

\usepackage{enumerate} % 列表

\lstset{
    %backgroundcolor=\color{red!50!green!50!blue!50},%代码块背景色为浅灰色
    rulesepcolor= \color{gray}, %代码块边框颜色
    breaklines=true,  %代码过长则换行
    numbers=left, %行号在左侧显示
    numberstyle= \small,%行号字体
    %keywordstyle= \color{red},%关键字颜色
    commentstyle=\color{gray}, %注释颜色
    frame=shadowbox%用方框框住代码块
    }

%\usepackage{url} % 引用URL
% \usepackage{cite}
% \newcommand{\upcite}[1]{\textsuperscript{\textsuperscript{\cite{#1}}}} %参考文献上标
%\bibliographystyle{plain}   %引用的样式%

\pagestyle{fancy}
\fancyhf{} 
\cfoot{\thepage} 

\setlength\parindent{0pt} % Removes all indentation from paragraphs

\renewcommand{\labelenumi}{\alph{enumi}.} 

%----------------------------------------------------------------------------------------
%	DOCUMENT INFORMATION
%----------------------------------------------------------------------------------------
\title{算法分析与设计-作业九}

\author{王宸昊 2019214541}

\date{\today}

\begin{document}

\maketitle

%----------------------------------------------------------------------------------------
%	SECTION 1
%----------------------------------------------------------------------------------------

\section{CLRS, Page, 428 26.2-10}

假设我们已经找到了最大流f, 假设在图G中边(u, v)的容量为f(u, v)。执行Ford-Fullkerson算法时,当某条边的流量达到其容量时移除此边,即在残差网络中不会出现回边。由最大流最小切割定理可知,对于其中每个增光路径,使其容量最小的边饱和,则条边一定是最小割的一部分。因为最多有|E|条增广路径,因此一定能在|E|次找到最大流的增广路径序列。


%----------------------------------------------------------------------------------------
%	SECTION 2
%----------------------------------------------------------------------------------------

\section{CLRS, Page, 446 26.5-4}

每次对高度溢出最高的顶点进行DISCHARGE操作:假设顶点V进入优先级队列,一定存在一个顶点u进行PUSH(u,v)操作,由引理26.27可知,进行PUSH操作不会创建任何新的许可边。所以只要对节点V进行RELABEL操作之后,才能将其再次添加到优先级队列中。此外每一次的DISCHARGE都以PUSH操作结束,都会产生一个新的最高溢出结点因此每一个调用DISCHARGE操作都会有某一个顶点,直到RELABEL后才能添加该顶点到队列中。\\
进行|V|次DISCHARGE并且没有RELABEL操作之后,无法再进行PUSH操作,这是算法有两种情况,一种是算法终止,或者执行一次RELABEL操作。所以如此循环下去,一定能在$O(V^3)$次DISCHARGE操作之后,算法结束。\\
因此每次先释放高度最高的溢出结点时,算法的复杂度为$O(V^3)$


%----------------------------------------------------------------------------------------
%	SECTION 3
%----------------------------------------------------------------------------------------
\section{CLRS, Page, 447 26-2}

\subsection{算法思想:}

如下面的提示,构建图G',将图G'中每条边的容量设置为1,运行最大流算法。如果$(x_i, y_i)$存在最大流中存在为1的流,添加(i,j)是路径覆盖中的一条边,其中最小的路径覆盖就是最小路径覆盖。即将每个点x拆为$x_1, y_1$,前者作为二分图的左边点,后者作为二分图的右边点,通过建立超级源点和超级汇点,使用最大流算法,则最小路径覆盖的数目为顶点数目-最大流\\

首先证明:在同一个路径集合中,每个顶点只出现一次。\\
有反证法可知,假如存在一个路径集合中有重复顶点,则对于$i \neq k \neq j$,存在$f(x_i, y_j) = f(x_k, y_j)$,因为$y_j$的流出流量为1,所以上面的假设不成立。\\

此外,因为从s到$x_i$的容量为1,$k \neq j$时,不存在$(x_i, y_j)$和$(x_i, y_k)$形式的边,我们可以通过判断$(x_i, y_j)$没有边,来保证在每个顶点都在某个路径中。因此我们可以保证获得一条覆盖路径。\\

如果在n个顶点的覆盖中有k条路径,则它们将总共由n-k条边组成。假设有一条覆盖路径,当且仅当$(i,j)$是覆盖路径当中的一条边时,可以通过分配边$(x_i,x_j)$的流为1来恢复流。假设最大流量算法产生一个覆盖k条路径的覆盖,因此流量为n-k,但是最小路径覆盖所使用的严格少于k条路径,它必须严格使用n-k个以上的边,因此我们可以恢复的流大于先前找到的流,这与先前最大流相矛盾。因此我们可以找到一条最小路径覆盖。

\subsection{是否可以应用于带环路的有向图:}

不可以用于带环路的有向图,假设图G有顶点(1,2,3,4),其存在边(1,2)(2,3)(3,1)(4,3),正确的输出为只有一条:4,3,1,2。\\
但是上述算法中重新分配后的边$(x_1,y_2), (x_2,y_3)(x_3, y_1)$最大流为1,与正确答案不符。

\end{document}
