%%%%%%%%%%%%%%%%%%%%%%%%%%%%%%%%%%%%%%%%%
% University/School Laboratory Report
% LaTeX Template
% Version 3.1 (25/3/14)
%
% This template has been downloaded from:
% http://www.LaTeXTemplates.com
%
% Original author:
% Linux and Unix Users Group at Virginia Tech Wiki 
% (https://vtluug.org/wiki/Example_LaTeX_chem_lab_report)
%
% License:
% CC BY-NC-SA 3.0 (http://creativecommons.org/licenses/by-nc-sa/3.0/)
%
%%%%%%%%%%%%%%%%%%%%%%%%%%%%%%%%%%%%%%%%%

%----------------------------------------------------------------------------------------
%	PACKAGES AND DOCUMENT CONFIGURATIONS
%----------------------------------------------------------------------------------------

\documentclass[UTF8]{ctexart}

\usepackage{siunitx} % Provides the \SI{}{} and \si{} command for typesetting SI units
\usepackage{graphicx} % Required for the inclusion of images
\usepackage{graphics}%图片设置
\usepackage{subfigure} 

\usepackage{natbib} % Required to change bibliography style to APA
\usepackage{amsmath} % Required for some math elements 
\usepackage{amssymb} % 使用因为所以符号
\usepackage{fancyhdr} % 使用页眉

\usepackage{algorithm}
\usepackage{algorithmic}

\usepackage{listings} % 插入代码
\usepackage{xcolor}
\lstset{
    %backgroundcolor=\color{red!50!green!50!blue!50},%代码块背景色为浅灰色
    rulesepcolor= \color{gray}, %代码块边框颜色
    breaklines=true,  %代码过长则换行
    numbers=left, %行号在左侧显示
    numberstyle= \small,%行号字体
    %keywordstyle= \color{red},%关键字颜色
    commentstyle=\color{gray}, %注释颜色
    frame=shadowbox%用方框框住代码块
    }

%\usepackage{url} % 引用URL
% \usepackage{cite}
% \newcommand{\upcite}[1]{\textsuperscript{\textsuperscript{\cite{#1}}}} %参考文献上标

\pagestyle{fancy}
\fancyhf{} 
\cfoot{\thepage} 

\setlength\parindent{0pt} % Removes all indentation from paragraphs

\renewcommand{\labelenumi}{\alph{enumi}.} 

%----------------------------------------------------------------------------------------
%	DOCUMENT INFORMATION
%----------------------------------------------------------------------------------------
\title{算法分析与设计-作业四}

\author{王宸昊 2019214541}

\date{\today}

\begin{document}

\maketitle

%----------------------------------------------------------------------------------------
%	SECTION 1
%----------------------------------------------------------------------------------------

\section{CLRS, Page, 210 15.1-4}

\subsection{实现思路}

使用s[i]来记录切割点的坐标,在每次更新状态值时,记录相应的位置信息。\\
递归结束后,依次寻找切分点。\\

\subsection{实现代码}

\begin{lstlisting}[language={python}]
    # 动态规划-带备忘录的自顶向下方法-记录路径
    def Extended_Memoized_Cut_Rod(p, n):
        r = [-1] * (n + 1)
        s = [0] * (n + 1)  # 记录最优解路径
        Extended_Memoized_Cut_Rod_Aux(p, n, r, s)
        print(r)
        print(s)
        print("最优解:%d" % r[n])
        print("最优路线:")
        while n > 0:
            print(s[n])
            n = n - s[n]

    def Extended_Memoized_Cut_Rod_Aux(p, n, r, s):
        if r[n] >= 0:
            return (r[n])
        if n == 0:
            max_value = 0
        else:
            max_value = -1
        for i in range(1, n+1):
            temp = Extended_Memoized_Cut_Rod_Aux(p, n-i, r, s)
            if max_value < p[i] + temp:
                max_value = p[i] + temp
                s[n] = i    # 记录最优时切分点的位置
        r[n] = max_value    # 记录下n时的最小值
        
        return max_value
\end{lstlisting}

\subsection{输出结果}

\begin{figure}[H]
    \centering
    \includegraphics[width=1\textwidth]{img/res-1.png}
    \caption{不同规模下排序算法时间对比}
    \label{带切割方案的备忘录方法}
\end{figure}


%----------------------------------------------------------------------------------------
%	SECTION 2
%----------------------------------------------------------------------------------------

\section{CLRS, Page, 114 8.4-4}

证明:\\
借鉴桶排序的思想,关键问题将点的分布均匀的映射到不同的桶中。因为所有的点服从均匀分布,所以将半径为1的圆,根据面积等分为n份,即等分为n个面积为1/n的圆环,则第i个圆盘的半径满足:

\begin{align*}
    \pi({r_i}^2-{r_{i-1}}^2) &= \frac{\pi}{n} \\
    {r_i}^2-{r_{i-1}}^2 &= \frac{1}{n}\\
    \vdots\\
    {r_2}^2-{r_{1}}^2 &= \frac{1}{n}
\end{align*}

因此,由上式的递推公式可得,第i个圆环的半径为$r_i=\sqrt{\frac{i}{n}}$。\\
则对应桶的半径区间为$[0,\sqrt{\frac{1}{n}}), [\sqrt{\frac{i}{n}}, \sqrt{\frac{2}{n}}), \ddots, [\sqrt{\frac{n-1}{n}}, 1) $。

由上所知共有n个桶,根据下面公式,即可由每个点的d得到应放在第k个桶内:

\begin{equation*}
    k=\left\{
    \begin{array}{rcl}
    \left\lfloor d * n\right\rfloor + 1 & & {d < 1}\\
    n & & {d = 1}
    \end{array} \right.
\end{equation*}
  
%----------------------------------------------------------------------------------------
%	SECTION 3
%----------------------------------------------------------------------------------------

\section{对比不同排序算法的排序效果}

\subsection{算法简介}

\textbf{插入排序:}\\


\subsection{算法实现}

算法的具体实现在src目录下的sort.py当中,使用的语言为python,在程序中分别封装了5中算法的具体实现,在main函数中生成不同规模的随机数据,在这里随机算法选用Python的random()模块,可以产生任意大小的随机数。需要注意的是,为了消除待排序数组的分布对排序结果的影响,对于某一数据规模下的实验,我们对不同算法采用相同的测试集,然后将结果输出到data目录下的data.txt当中。

\subsection{实验环境}
编译环境为Python3.7,操作系统的版本为Windows 10。\\
CPU为AMD Ryzen 7 1700(3.0 GHz),RAM 16G。\\
第三方模块:
\begin{itemize}
\item time(): 对算法执行过程进行高精度的计时,最高精度可达微秒级别
\item random(): 产生任意大小的随机数
\item sys():查看内存使用情况
\end{itemize}

\subsection{实验结果}


\begin{table}[H]
    \caption{$10^9$输入规模下算法性能测试}
    \label{table-1}
    \begin{center}
        \begin{tabular}{cc}
            \hline
            排序算法&   执行时间(s)\\     
            \hline
            快速排序&       13510.133\\               
            归并排序&       17861.703\\              
            希尔排序&       23620.192\\             
            基数排序&      9998.141\\                      
            \hline
        \end{tabular}  
    \end{center}
\end{table}

从上表的数据可以明显的看出,从总体上看,在数据量比较大的时候,排序是一件相当耗时的事情。以快排为例,进行一次排序约3.7个小时。同时可以看出,在基于比较的算法的当中,快速排序的算法的执行时间是最优秀的,基数排序作为一种接近线性的算法,其执行时间明显优于其他基于比较的算法,在10进制的情况下,只需要2.7个小时就可以完成排序。

\end{document}