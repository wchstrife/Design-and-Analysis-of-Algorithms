%%%%%%%%%%%%%%%%%%%%%%%%%%%%%%%%%%%%%%%%%
% University/School Laboratory Report
% LaTeX Template
% Version 3.1 (25/3/14)
%
% This template has been downloaded from:
% http://www.LaTeXTemplates.com
%
% Original author:
% Linux and Unix Users Group at Virginia Tech Wiki 
% (https://vtluug.org/wiki/Example_LaTeX_chem_lab_report)
%
% License:
% CC BY-NC-SA 3.0 (http://creativecommons.org/licenses/by-nc-sa/3.0/)
%
%%%%%%%%%%%%%%%%%%%%%%%%%%%%%%%%%%%%%%%%%

%----------------------------------------------------------------------------------------
%	PACKAGES AND DOCUMENT CONFIGURATIONS
%----------------------------------------------------------------------------------------

\documentclass[UTF8]{ctexart}

\usepackage{siunitx} % Provides the \SI{}{} and \si{} command for typesetting SI units
\usepackage{graphicx} % Required for the inclusion of images
\usepackage{graphics} % 图片设置
\usepackage{subfigure} 

\usepackage{natbib} % Required to change bibliography style to APA
\usepackage{amsmath} % Required for some math elements 
\usepackage{amssymb} % 使用因为所以符号
\usepackage{fancyhdr} % 使用页眉

\usepackage{algorithm}
\usepackage{algorithmic}

\usepackage{listings} % 插入代码
\usepackage{xcolor}

\usepackage{enumerate} % 列表

\lstset{
    %backgroundcolor=\color{red!50!green!50!blue!50},%代码块背景色为浅灰色
    rulesepcolor= \color{gray}, %代码块边框颜色
    breaklines=true,  %代码过长则换行
    numbers=left, %行号在左侧显示
    numberstyle= \small,%行号字体
    %keywordstyle= \color{red},%关键字颜色
    commentstyle=\color{gray}, %注释颜色
    frame=shadowbox%用方框框住代码块
    }

%\usepackage{url} % 引用URL
% \usepackage{cite}
% \newcommand{\upcite}[1]{\textsuperscript{\textsuperscript{\cite{#1}}}} %参考文献上标
%\bibliographystyle{plain}   %引用的样式%

\pagestyle{fancy}
\fancyhf{} 
\cfoot{\thepage} 

\setlength\parindent{0pt} % Removes all indentation from paragraphs

\renewcommand{\labelenumi}{\alph{enumi}.} 

%----------------------------------------------------------------------------------------
%	DOCUMENT INFORMATION
%----------------------------------------------------------------------------------------
\title{算法分析与设计-作业八}

\author{王宸昊 2019214541}

\date{\today}

\begin{document}

\maketitle

%----------------------------------------------------------------------------------------
%	SECTION 1
%----------------------------------------------------------------------------------------

\section{CLRS, Page, 395 24-1}

\subsection{a. 证明无环和拓扑排序顺序}

$G_f$中的边两边的顶点下标是递增的,如果有环路的话,那么一定会出现顶点坐标递减,跟$G_f$边的定义相反,所以$G_f$是无环的。同理$G_b$也是无环的。

根据$G_f$的定义,其所有的顶点都是从低到高的,因此其拓扑排序的顺序就是$<V_1, V_2, \dots, V_{|V|}>$。同理$G_b$

\subsection{b. 证明松弛操作次数}

在最好的情况下,当每个顶点都以正序或倒序的顺序排列时,通过对$E_f$$E_b$中的边进行一次松弛操作就可以得到各个顶点的最短路径。

在最坏的情况下,顶点的顺序先增后减或者先减后增的顺序交替出现,所以整个路径中有$|V|-2$次顺序的改变,每两次顺序改变就要对所有的边进行松弛操作,因此一共进行$|V|/2$遍松弛操作。

\subsection{c. 是否改善了Bellman-Ford算法渐进时间}

并没有降低Bellan-Ford算法的渐进运行时间。只是降低了常数项,而整体的渐进复杂度还是O(VE)


%----------------------------------------------------------------------------------------
%	SECTION 2
%----------------------------------------------------------------------------------------

\section{CLRS, Page, 370 23.2-7}

\end{document}
